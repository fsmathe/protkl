% \iffalse
% prot package
% Copyright 2021 Markus Kurtz
%
% If called with nonLaTeX or with \install=y or as a batch file
% generate some nice files.  Otherwise produce documentation.
%<*ignore>
	{\def\x{LaTeX2e}\expandafter}%
	\ifcase0% if \install=y or \processbatchFile is defined or \fmtname=LaTeX2e
		\ifx\install y1\fi
		\expandafter\ifx\csname processbatchFile\endcsname\relax\else1\fi
		\ifx\fmtname\x\else1\fi
	\else\csname fi\endcsname% ... go on after here
%</ignore>
%<*install>
	\input docstrip.tex
	\askforoverwritefalse
	\keepsilent
	\preamble
	This is a prot file.
	\endpreamble
	\generate{
		\file{prot.ins}{\from{prot.dtx}{install}}
		\file{prot.drv}{\from{prot.dtx}{driver}}% for documentation
		\file{prot.sty}{\from{prot.dtx}{package}}
	}
	\usedir{testfiles}
	\generate{
		\file{prot.lvt}{\from{prot.dtx}{test}}
		\file{prot.lve}{\from{prot.dtx}{expect}}
	}
	\endbatchfile
%</install>
%<*ignore>
	\fi % end the \ifcase and produce documentation
%</ignore>
%<*driver>
	\documentclass{l3doc}
	% Allow to write |\cs|
	\MakeShortVerb{\|}
	% Index via codeline
	\CodelineIndex
	\RecordChanges
	\begin{document}
		\DocInput{prot.dtx}
\end{document}
%</driver>
%
% \fi  end the \iffalse
%
% \title{prot}
% \author{Markus Kurtz}
% \maketitle
% Dieses Paket dient dem Schreiben von Protokollen. Besonders erlaubt es, über die anwesenden Personen, Abstimmungsergebnisse, Aufgaben und Resultate Buch zu führen.
% \section{Layout}
% Wird großteils an \pkg{KoMa-Script} abgegeben.
% Für vorläufige Protokolle (Paket-Option) werden Zeilennummern ausgegeben.
% Lädt \pkg{hyperref} und setzt Titel und Autorin gemäß den Angaben.
% Angaben zu Ort und Zeit der Sitzung werden ebenfalls für Titel im Dokument und in den PDF-Informationen genutzt.
% Angaben zu Ort und Zeit der nächsten Sitzung werden am Ende ausgegeben.
% Am Ende des Protokolls bzw.\@ vor dem Anhang erscheint zudem eine Signatur.
% Es kann eine Begrüßungs- und Abschiedsfloskel ausgegeben werden.
% \section{Personen}
% Zu Beginn des Protokolls werden Personen mit dem Befehl \cs{neuePerson} angelegt, was typischerweise über \env{personenliste} geschieht.
% Für jede Person legt \cs{neuePerson} basierend auf den Argumenten eine \meta{id} fest, definiert \cs{\meta{id}} als einfaches Makro, was \meta{name} zurückgibt und speichert einige Daten.
% Zu Beginn des Dokuments kann dann eine Anwesenheitsliste oder -tabelle ausgegeben werden.
% Hiefür werden die Personen innerhalb ihrer \meta{group} sortiert und mit ggf.\@ nötiger Zusatzinformation gemeinsam ausgegeben.
% Später kann dann bei Abstimmungsergebnissen automatisch die Gesamtstimmzahl ermittelt werden und etwa mit |\erg*00| ein einstimmiges Abstimmungsergebnis ausgegeben werden.
% Hierfür ist es natürlich nötig, auch speziell zu notieren, wann Stimmberechtigte Personen die Sitzung betreten und verlassen.
% \section{Ergebnisübersicht}
% Alle Ergebnisse werden in eine ausgelagerte Datei geschrieben und können wahlweise im Protokoll selbst erscheinen oder auch in einem Übersichtsdokument oder sonstwo.
% \section{Textbausteine}
% Der Freundlichkeit halber werden verschiedene Textbausteine etwa für Währungs- und Zeitangaben bereitgestellt sowie für Einschübe für Sitzungspausen oder andere Meta-Informationen.
% Auch nützlich sind die Möglichkeiten, Befehle zu erstellen, die bei erstmaliger Verwendung eine Erklärung bereitstellen.
% \StopEventually{\PrintIndex\PrintChanges}
%
% \setcounter{StandardModuleDepth}{1}
%
% \section{Implementation}
%
% This is a prot and fun implementation.
% And here it starts.
% Nutzt \pkg{pgfkeys} statt \pkg{l3keys} oder \pkg{xkeyval}, um Werte speichern zu können und weil \pkg{pgfkeys} einfach generell mehr kann und eine buntere Dokumentation besitzt :D
%    \begin{macrocode}
%<@@=prot>
%<*test|expect>
	\input{regression-test}
	\RequirePackage{prot}
	\RequirePackage{expl3}
	\START
	\ExplSyntaxOn
	\keys_set:nn { sys } { check-declarations, log-functions }
%<*test>
	\cs_set:Nn\@@_test:nn{\typeout{#1}}
	\cs_set_eq:NN\@@_nontest:nn\use_i:nn
%</test>
%<*expect>
	\cs_set:Nn\@@_test:nn{\typeout{#2}}
	\cs_set_eq:NN\@@_nontest:nn\use_ii:nn
%</expect>
%</test|expect>
%<*package>
	\RequirePackage{expl3}
	\ProvidesExplPackage{prot}{2021-11-26}{1.0}{A prot and Fun Package}
	\RequirePackage{pgfkeys}
%    \end{macrocode}
% \begin{macro}{\prot_newperson_frontend:...}
%   Erstellt \meta{id} und \meta{name} mittels \cs{@@_set_idname_via_shorty:nnnn} und verarbeitet Angaben zur Gruppenzugehörigkeit, Anwesenheit und Zusatzinformationen.
%    \begin{macrocode}
	\str_new:N \l@@_id_str
	\tl_new:N \l@@_name_tl
	\cs_new:Npn \prot_newperson_frontend:nnnnnnn #1#2#3#4 {
		\@@_shorty:nnnn {#1} {#2} {#3} {#4}
		\exp_args:NVV \prot_newperson:nnnnnnnn \l@@_id_str \l@@_name_tl {#3} {#4}
	}
%    \end{macrocode}
% \end{macro}
% \begin{macro}{\@@_set_idname_via_shorty:nnnn}
%   Nimmt |id, first, last, shorty|
%   Legt basierend auf dem ersten Argument \marg{shorty} fest, wie der Name abgekürzt werden soll.
%   Dazu nutzt es \pkg{l3str} und \pkg{l3tl}.
%    \begin{macrocode}
	\cs_new:Nn \@@_set_idname_via_shorty:nnnn {
		\tl_if_blank:nTF {#4} {
			\@@_set_idname:n {#2}
		} {
			\str_set:Nn \l@@_id_str {#2}
			\tl_set:Nn \l@@_name_tl {#2}
			\str_case:onF { \tl_head:w #4 {}\q_stop } {
				{=} { \tl_put_right:Nn \l@@_name_tl { ~ #3 } }
				{?} { \tl_put_right:Nx \l@@_name_tl { ~ \tl_range:nnn {#4} {2} {-1} } }
				{:} { \tl_set:Nx \l@@_name_tl { \tl_range:nnn {#4} {2} {-1} } }
				{!} { \@@_set_abbreviation:x { \tl_range:nne {#3} {1} { \tl_count:n {#4} } } }
			} {
				\@@_set_idname:n {#4}
			}
		}
		\tl_if_blank:nF {#1} { \str_set:Nn \l@@_id_str {#1} }
		\str_remove_all:Nn \l@@_id_str {~}
		\str_remove_all:Nn \l@@_id_str {-}
	}
	\cs_generate_variant:Nn \tl_range:nnn { nne }
	\cs_new:Nn \@@_set_idname:n {
		\str_set:Nn \l@@_id_str {#1}
		\tl_set:Nn \l@@_name_tl {#1}
	}
	\cs_new:Nn \@@_set_abbreviation:n {
		\str_put_right:Nn \l@@_id_str {#1}
		\tl_put_right:Nn \l@@_name_tl { \c_prot_nbsp #1 . }
	}
	\cs_generate_variant:Nn \@@_set_abbreviation:n { x }
	\tl_const:Nn \c_prot_nbsp { \penalty10000\  }
%</package>
%<*test|expect>
	\TEST { \@@_set_idname_via_shorty:nnnn } {
%<*test>
	\cs_new:Nn \@@_test_shorty:nnnnnn {
		\@@_set_idname_via_shorty:nnnn {#3} { #1 } { #2 } {#4}
		\TYPE { { \l@@_id_str } ~ { \l@@_name_tl } }
	}
%</test>
%<*expect>
	\cs_new:Nn \@@_test_shorty:nnnnnn { \TYPE { {#5} ~ {#6} } }
%</expect>
	\cs_new:Npn \@@_test_donald:nnnn { \@@_test_shorty:nnnnnn { Donald } { Knuth } }
	\@@_test_donald:nnnn {} {} { Donald } { Donald }
	\@@_test_donald:nnnn {} { = } { Donald } { Donald ~ Knuth }
	\@@_test_donald:nnnn {} { ?Duck } { Donald } { Donald ~ Duck }
	\@@_test_donald:nnnn {} { :Don } { Donald } { Don }
	\@@_test_donald:nnnn {} { ! } { DonaldK } { Donald \c_prot_nbsp K. }
	\@@_test_donald:nnnn {} { !! } { DonaldKn } { Donald \c_prot_nbsp Kn. }
	% The following is weird, but you asked for it :D
	\@@_test_donald:nnnn {} { !23456 } { DonaldKnuth } { Donald \c_prot_nbsp Knuth. }
	\@@_test_donald:nnnn {} { Don } { Don } { Don }
	\@@_test_donald:nnnn { DK } { Don } { DK } { Don }
	\cs_new:Npn \@@_test_anna:nnnn { \@@_test_shorty:nnnnnn { Anna-Lena } { Rot ~ Grau-Schwarz } }
	\@@_test_anna:nnnn {} {} { AnnaLena } { Anna-Lena }
	\@@_test_anna:nnnn {} { !!! } { AnnaLenaRot } { Anna-Lena \c_prot_nbsp Rot . }
	\@@_test_anna:nnnn {} { !!!! } { AnnaLenaRotG } { Anna-Lena \c_prot_nbsp Rot ~ G . }
	\@@_test_anna:nnnn {} { !otGrau- } { AnnaLenaRotGrau } { Anna-Lena \c_prot_nbsp Rot ~ Grau- . }
	\@@_test_anna:nnnn {} { !otGrau-S } { AnnaLenaRotGrauS } { Anna-Lena \c_prot_nbsp Rot ~ Grau-S . }
	}
%</test|expect>
%    \end{macrocode}
% \end{macro}
% \begin{macro}{\@@_parse_attendance:n, \@@_set_attendance:n, \@@_apply_group:n}
%   Über das Anwesenheitsfeld wird die Anwesenheit als ein Single-Choice-Wert definiert,
%   üblicherweise eins aus: x (anwesend), d (digital), e (entschuldigt), u (unentschuldigt).
%   Zusätzlich können hier noch weitere Attribute festgelegt werden: p (Protokoll), s (Sitzungsleitung)
%   beide setzen den Säumniswert für die Anwesenheit auf „x“ und rufen |\...| auf, was in der Person
%   einen Schalter setzt und die Person in der Liste für Protokoll/Sitzungsleitung einträgt.
%
%   Verschiedene Anwesenheitsmarker können als Schlüssel unter \emph{/prot/attendance/} angelegt werden.
%   Jede Gruppe ist implementiert als Menge an Personen. Zudem ist unter
%   \texttt{/prot/person/\meta{id}/group} die primäre Gruppe hinterlegt, wobei spätere Angaben vorherige überschreiben.
%
%   TODO: Implementiere mehrere Zeilen für eine Person.
%
%    \begin{macrocode}
%<*package>
	\cs_new:Nn \@@_parse_attendance:n {
		\pgfqkeys { /prot/attendance } { value/.link=/prot/attendance/fallback, fallback=u, specials=, #1 }
	}
	\cs_new:Nn \@@_set_attendance:n {
		\pgfqkeys { /prot/person/#1 } {
			attendance/.initial=, there/.initial=2, attendance/disp/.link=/prot/person/#1/attendance,
			/prot/attendance/use/\pgfkeysvalueof{/prot/attendance/value}
		}
	}
	\pgfkeys { /prot/attendance/.cd
		, value/.link = fallback
		, fallback/.initial = u
		, specials/.initial =
		, x/.style = {value=x}, use/x/.style = {attendance=x, there=1}
		, e/.style = {value=e}, use/e/.style = {attendance=e, there=0}
		, u/.style = {value=u}, use/u/.style = {attendance=u, there=0}
		, d/.style = {value=d}, use/d/.style = {attendance=d, there=1, attendance/disp=\emph x}
		, p/.style = {fallback=x, specials/.append=p}
		, s/.style = {fallback=x, specials/.append=s}
	}
%</package>
%<*test|expect>
\TEST { \@@_parse_attendance:n } {
%<*test>
	\cs_new:Nn \@@_test_attendance_i:nn {
		\@@_parse_attendance:n {#1}
		\pgfkeys { /prot/attendance/value/.get=\l_tmpa_tl } \TYPE {\l_tmpa_tl}
	}
	\cs_new:Nn \@@_test_attendance:nnnnn {
		\@@_parse_attendance:n {#1}
		\@@_set_attendance:n {Donald}
		\pgfkeys { /prot/person/Donald/attendance/.get=\l_tmpa_tl } \TYPE {attendance = \l_tmpa_tl}
		\pgfkeys { /prot/person/Donald/there/.get=\l_tmpa_tl } \TYPE {there = \l_tmpa_tl}
		\pgfkeys { /prot/person/Donald/attendance/disp/.get=\l_tmpa_tl } \TYPE { disp = #4 \l_tmpa_tl }
	}
%</test>
%<*expect>
	\cs_new:Nn \@@_test_attendance_i:nn { \TYPE {#2} }
	\cs_new:Nn \@@_test_attendance:nnnnn { \TYPE {attendance=#2} \TYPE {there=#3} \TYPE {disp=#5} }
%</expect>
	\@@_test_attendance_i:nn { x } { x }
	\@@_test_attendance_i:nn { p } { x }
	\@@_test_attendance_i:nn { u } { u }
	\@@_test_attendance_i:nn { d } { d }
	\@@_test_attendance_i:nn { x, e } { e }
	\@@_test_attendance_i:nn { e, x } { x }
	\@@_test_attendance:nnnnn { x } x 1 {} { x }
	\@@_test_attendance:nnnnn { p } x 1 {} { x }
	\@@_test_attendance:nnnnn { u } u 0 {} { u }
	\@@_test_attendance:nnnnn { d } d 1 {\exp_not:V} { \exp_not:N\emph x }
	\@@_test_attendance:nnnnn { x, e } e 0 {} { e }
	\@@_test_attendance:nnnnn { e, x } x 1 {} { x }
}
%</test|expect>
%<*package>
	\cs_new:Nn \@@_apply_group:n {
		\pgfkeys { /prot/person/\l@@_id_str/.cd, group/.initial=, info/.initial=, aftermove/.code=, /prot/group/.cd, #1 }
		% In Gruppen abspeichern
	}
	\pgfkeys {
		, /prot/voter/.style = {
			aftermove/.code = { \@@_movevoter:n ##1 }
		}
		, /prot/groupvoter/.style = {
			, aftermove/.code = { \@@_movefrom:nn {#1} ##1 }
		}
		, /prot/group/.code = {
			\pgfkeysalso { group = #1 }
			\seq_gput_right:cV { g_prot_group_#1_seq } \l@@_id_str
		}
		, /prot/group/.cd
		, g/.code = \pgfkeys { /prot/person/\l@@_id_str/.cd, /prot/group=gast, info=#1 }
		% StuPa
		, stupa/.code = \pgfkeys { /prot/person/\l@@_id_str/.cd, /prot/group=stupa, /prot/voter, info=#1 }
		, fsma/.code = \pgfkeys { /prot/person/\l@@_id_str/.cd, /prot/group=fsma, info=#1 }
		, reffs/.code = \pgfkeys { /prot/person/\l@@_id_str/.cd, /prot/group=reffs, info=#1 }
		% AStA
		, refvor/.code = \pgfkeys { /prot/person/\l@@_id_str/.cd, /prot/group=refvor, /prot/groupvoter=refvor, info=#1 }
	}
	\seq_new:N \g_prot_group_gast_seq
	\seq_new:N \g_prot_group_stupa_seq
	\seq_new:N \g_prot_group_fsma_seq
	\seq_new:N \g_prot_group_reffs_seq
	\seq_new:N \g_prot_group_refvor_seq
	\seq_new:N \g_prot_votergroup_refvor_seq
	\int_new:N \g_prot_voters_int
	\cs_new:Nn \@@_movevoter:n {
		\str_case_e:nnF {#1} {
			0 { \int_gdecr:N \g_prot_voters_int }
			1 { \int_gincr:N \g_prot_voters_int }
		} {
			\msg_error_text:n { movevoter }
		}
	}
	\cs_new:Nn \@@_movefrom:nn {
		\pgfkeysalso{id/.get=\l@@_id_str}
		\str_case_e:nnF {#2} {
			0 \@@_leavefrom:cV
			1 \@@_enterfrom:cV
		} {
			\msg_error_text:n { wrong-move } \use_none:nn
		}
		{ g_prot_votergroup_#1_seq } \l@@_id_str
	}
	\cs_new:Nn \@@_leavefrom:Nn {
		\seq_if_in:NnTF #1 {#2} {
			\seq_gremove_all:Nn #1 {#2}
			\seq_if_empty:NT #1 {
				\int_gdecr:N \g_prot_voters_int
			}
		} {
			\msg_error_text:n { out-of-nothing }
		}
	}
	\cs_new:Nn \@@_enterfrom:Nn {
		\seq_if_in:NnTF #1 {#2} {
			\msg_error_text:n { out-of-nothing }
		} {
			\seq_if_empty:NT #1 {
				\int_gincr:N \g_prot_voters_int
			}
			\seq_gput_right:Nn #1 {#2}
		}
	}
	\cs_generate_variant:Nn \@@_leavefrom:Nn {cV}
	\cs_generate_variant:Nn \@@_enterfrom:Nn {cV}
%</package>
%<*test|expect>
\TEST { \@@_apply_group:n } {
	\setcounter{errorcontextlines}{100}
	\cs_new:Npn \@@_test_seq_use:c #1 { \seq_use:cn {#1} , }
	\cs_new:Npn \@@_test_use_if_exist:Nc #1#2 { \cs_if_exist:cT {#2} { #2 = #1 { #2 } } }
%<*test>
	\cs_new:Nn \@@_test_group:nnnnnnn {
		\str_set:Nn \l@@_id_str {#1}
		\pgfkeys { /prot/person/#1/id/.initial = #1 }
		\pgfkeys { /prot/person/#1/there/.initial = #2 }
		\@@_apply_group:n {#3}
		\pgfkeys { /prot/person/#1/group/.get=\l_tmpa_tl }
		\TYPE { value = \l_tmpa_tl }
		\TYPE { \@@_test_use_if_exist:Nc \@@_test_seq_use:c { g_prot_group_#4_seq } }
		\@@_inimove:n {#1}
		\TYPE { \#voters = \int_use:N \g_prot_voters_int }
		\TYPE { \@@_test_use_if_exist:Nc \@@_test_seq_use:c { g_prot_votergroup_#4_seq } }
	}
	\cs_new:Nn \@@_test_move:nnnnn {
		\pgfkeys { /prot/person/#1/move = #2 }
		\pgfkeys { /prot/person/#1/there/.get=\l_tmpa_tl }
		\TYPE { there = \l_tmpa_tl }
		\TYPE { \#voters = \int_use:N \g_prot_voters_int }
		\TYPE { \@@_test_use_if_exist:Nc \@@_test_seq_use:c { g_prot_votergroup_#3_seq } }
	}
%</test>
%<*expect>
	\cs_new:Nn \@@_test_group:nnnnnnn {
		\TYPE { value = #4 }
		\TYPE { \quark_if_nil:nF {#5} { g_prot_group_#4_seq = #5 } }
		\TYPE { \#voters = #6 }
		\TYPE { \quark_if_nil:nF {#7} { g_prot_votergroup_#4_seq = #7 } }
	}
	\cs_new:Nn \@@_test_move:nnnnn {
		\TYPE { there = #2 }
		\TYPE { \#voters = #4 }
		\TYPE { \tl_if_empty:nF {#3} { g_prot_votergroup_#3_seq = #5 } }
	}
%</expect>
	\@@_test_group:nnnnnnn { Donald  } 0 { g }      { gast }  { Donald } 0 { \q_nil }
	\@@_test_group:nnnnnnn { Donald  } 1 { g }      { gast }  { Donald , Donald } 0 { \q_nil }
	\SEPARATOR
	\@@_test_group:nnnnnnn { Anna    } 0 { stupa }  { stupa } { Anna } 0 { \q_nil }
	\@@_test_group:nnnnnnn { Bea     } 1 { stupa }  { stupa } { Anna , Bea } 1 { \q_nil }
	\@@_test_group:nnnnnnn { Charlie } 1 { stupa }  { stupa } { Anna , Bea , Charlie } 2 { \q_nil }
	\@@_test_group:nnnnnnn { Dora    } 1 { stupa }  { stupa } { Anna , Bea , Charlie , Dora } 3 { \q_nil }
	\SEPARATOR
	\@@_nontest:nn { \@@_makelate:n { Bea } } { }
	\@@_test_move:nnnnn    { Donald  } 0 { } 2 { }
	\@@_test_move:nnnnn    { Bea     } 1 { } 3 { }
	\@@_test_move:nnnnn    { Charlie } 0 { } 2 { }
	\@@_test_move:nnnnn    { Bea     } 0 { } 1 { }
	\@@_test_move:nnnnn    { Dora    } 0 { } 0 { }
	\SEPARATOR
	\@@_test_group:nnnnnnn { Emilie } 0 { refvor }  { refvor } { Emilie } 0 { }
	\@@_test_group:nnnnnnn { Fynn   } 1 { refvor }  { refvor } { Emilie , Fynn } 1 { Fynn }
	\@@_test_group:nnnnnnn { Gerda  } 1 { refvor }  { refvor } { Emilie , Fynn , Gerda } 1 { Fynn , Gerda }
	\@@_test_group:nnnnnnn { Heidi  } 1 { refvor }  { refvor } { Emilie , Fynn , Gerda , Heidi } 1 { Fynn , Gerda , Heidi }
	\SEPARATOR
	\@@_nontest:nn { \@@_makelate:n { Gerda } }
	\@@_nontest:nn { \@@_makelate:n { Heidi } }
	\@@_test_move:nnnnn    { Donald } 1 { refvor } 1 { Fynn }
	\@@_test_move:nnnnn    { Heidi  } 1 { refvor } 1 { Fynn , Heidi }
	\@@_test_move:nnnnn    { Fynn   } 0 { refvor } 1 { Heidi }
	\@@_test_move:nnnnn    { Gerda  } 1 { refvor } 1 { Heidi , Gerda }
	\@@_test_move:nnnnn    { Gerda  } 0 { refvor } 1 { Heidi }
	\@@_test_move:nnnnn    { Heidi  } 0 { refvor } 0 { }
}
%</test|expect>
%    \end{macrocode}
% \end{macro}
% \begin{macro}{\prot_newperson:nnnnnnnn}  \prot_newperson_frontend:... produces input for that one by parsing the \meta{shortname} argument.
%   Nimmt \marg{id}, \marg{name}, \marg{first_name}, \marg{last_name}, \marg{group}, \marg{attendance}, \marg{special}, \marg{voter}.
%   Zuvor müssen diese bereits befüllt werden.
%   Dabei werden \marg{id} und \marg{name} von \prot_newperson_fancy:... und \marg{group} und \marg{voter} protokollspezifisch festgelegt.
%   \item Der Anzeigename \meta{name} wird anhand des \meta{shortname}-Arguments festgelegt.
%   \item Die \meta{id} wird falls vorhanden und nichtleer durch das \meta{id}-Argument und andernfalls gleich zu \meta{name} festgelegt.
%   Sollte \cs{\meta{id}} bereits definiert sein, wird ein Fehler erzeugt und das Anlegen der Person abgebrochen. Ansonsten wird \cs{\meta{id}} als \meta{name} mit folgendem |\@\xspace| definiert.
%   \NB{Das |\@| sorgt dafür, dass Abstände wie bei einem normalen Wortende (also einem Kleinbuchstaben) gesetzt werden, auch wenn der Name möglicherweise mit einem Punkt oder Großbuchstaben enden sollte.}
%
%   Zuerst wird mittels \pkg{pgfkeys} unter \texttt{/prot/names/\meta{name}} die \meta{id} hinterlegt, und dabei geschaut, ob \meta{name} bereits verwendet wird,
%       um einen Fehler inklusive Verweis auf die andere Person zu generieren.
%   \end{enumerate}
%    \begin{macrocode}
%<*package>
	\cs_new:Nn\prot_newperson:nnnnnnnn{
		\cs_if_exists:cTF{#1}{
			\pgfkeysifdefined{/prot/person/#1}
				{ \msg_error_text:n{prot/id} }
				{ \msg_error_text:n{prot/id} }
		}{
			\cs_new:cpn{#1}{#2\@\xspace}
			\pgfkeysifdefined{/prot/names/#2}
				{ \msg_error_text:n{/prot/names} }
				{ \pgfkeyssetvalue{/prot/names/#2}{#1} }
%    \end{macrocode}
%   Danach werden unter \texttt{/prot/person/\meta{id}/} verschiedene Schlüssel befüllt
%   (oder vielleicht auch nicht alle, wenn aus anderen bestimmbar):
%   |name|, |name/full|, |name/first|, |name/last|, |name/sort|, |attendance|, |group|, |special|, |voter|.
%    \begin{macrocode}
			\pgfkeyssetvalue{/prot/person/#1/id}{#1}
			\pgfkeyssetvalue{/prot/person/#1/name}{#2}
			\pgfkeyssetvalue{/prot/person/#1/name/first}{#3}
			\pgfkeyssetvalue{/prot/person/#1/name/last}{#4}
			\@@_parse_attendance:n {#6}
			\@@_set_attendance:n {#1}
			\@@_apply_group:n {#5}
			\@@_inimove:n {#1}
			\pgfkeyssetvalue{/prot/person/#1/special}{#7}
			\pgfkeyssetvalue{/prot/person/#1/voter}{#8}
		}
	}
%    \end{macrocode}
% \end{macro}
%   Womöglich sollte auch noch unter \texttt{prot/group/} hinterlegt werden, wer zu welcher Gruppe gehört, insbesondere wenn eine Gruppe kollektiv stimmberechtigt ist.
% \begin{macro}{\@@_move:nn}
%   Momentane Implementierung: Alle Menschen kommen automatisch, wenn sie mit |x| oder dergleichen angelegt sind.
%   Beim Einlesen der |.aux|-Datei gehen alle, deren erste Interaktion ein Kommen ist.
%   Bei allen späteren Bewegungen prüfe, ob dies die erste Bewegung ist und falls diese ein Kommen ist, schreibe in die |.aux|-Datei ein (nicht als erste Bewegung zählendes Gehen (siehe vorheriger Punkt)).
%   Bei einer nicht-ersten Bewegung hingegen verifiziere, dass diese das Gegenteil der vorherigen Anwesenheit ist.
%   Momentane Anwesenheit von #1=\meta{id} auf |#2| setzen.
%   Ruft hauptsächlich |attendance/move| auf, was beim ersten Aufruf in die |.aux|-Datei schreibt und bei späteren Aufrufen auf Konsistenz prüft.
%   Nutzt |forcemove| ohne Tests, |firstmove| und |latermove| sowie – als Hook – |aftermove|.
%   \NB{Beim ersten Kompilieren produziert |firstmove| Schrott, da die Verspäteten dann zweimal hintereinander (bei Initialisierung und jetzt) kommen. Könnte man beheben, aber wozu. Bei späterem Kompilieren darf |firstmove| korrekterweise immer |forcemove| aufrufen.}
%    \begin{macrocode}
	\cs_new:Nn \@@_inimove:n {
		\tl_set:Nx \l_tmpa_tl { \pgfkeysvalueof{/prot/person/#1/there} }
		\tl_if_eq:NnTF \l_tmpa_tl 1 {
			\pgfkeys { /prot/person/#1/move/.code = {\@@_move:nN {#1} {##1} } }
			\@@_forcemove:nN {#1} 1
		} {
			\pgfkeys { /prot/person/#1/move/.code = {\msg_error_text:n {not-there-move} } }
		}
	}
	\cs_new:Nn \@@_forcemove:nN {
		\pgfkeys { /prot/person/#1/.cd
			, there = #2
			, aftermove = #2
		}
	}
	\cs_new:Nn \@@_move:nN {
		\@@_firstmove:nN {#1} {#2}
		\pgfkeys { /prot/person/#1/move/.code = {\@@_latermove:nN {#1} {##1} } }
	}
	\cs_new:Nn \@@_firstmove:nN {
		\tl_if_eq:NnT {#2} 1 { \@@_addlate:n {#1} }
		\@@_forcemove:nN {#1} {#2}
	}
	\cs_new:Nn \@@_latermove:nN {
		\tl_set:Nn \l_tmpa_tl { \pgfkeysvalueof{/prot/person/#1/there} }
		\tl_if_eq:NnTF \l_tmpa_tl {#2}
			{ \msg_error_text:n { #2-again } }
			{ \@@_forcemove:nN {#1} {#2} }
	}
	\cs_new:Nn \@@_addlate:n {
		\iow_now:Nn \@mainaux { \@@_makelate:n {#1} }
	}
	\cs_new:Nn \@@_makelate:n {
		\@@_forcemove:nN {#1} 0
	}
%    \end{macrocode}
% And the end.
%    \begin{macrocode}
%</package>
%<*test|expect>
	\ExplSyntaxOff
	\END
%</test|expect>
%    \end{macrocode}
%
% \Finale
\endinput
